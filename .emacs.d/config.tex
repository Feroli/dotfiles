% Created 2016-02-25 Thu 10:26
\documentclass[11pt]{article}
\usepackage[utf8]{inputenc}
\usepackage[T1]{fontenc}
\usepackage{fixltx2e}
\usepackage{graphicx}
\usepackage{longtable}
\usepackage{float}
\usepackage{wrapfig}
\usepackage{rotating}
\usepackage[normalem]{ulem}
\usepackage{amsmath}
\usepackage{textcomp}
\usepackage{marvosym}
\usepackage{wasysym}
\usepackage{amssymb}
\usepackage{hyperref}
\tolerance=1000
\author{Fernando}
\date{\today}
\title{Emacs Configuration}
\hypersetup{
  pdfkeywords={},
  pdfsubject={},
  pdfcreator={Emacs 25.0.91.1 (Org mode 8.2.10)}}
\begin{document}

\maketitle
\tableofcontents

This emacs configuration is largely influenced by \href{https://github.com/danielmai/.emacs.d}{Mai's emacs configuration}

\section{Installing Emacs}
\label{sec-1}

On OS X, \href{http://brew.sh/}{homebrew} is an easy way to install emacs.

Here's what the parameter means.
\begin{description}
\item[{\verb~--with-cocoa~}] installs emacs with the OS X GUI version
\item[{\verb~--with-imagemagick~}] installs emacs with imagemagick support for image processing
\item[{\verb~--with-gnutls~}] installs emacs with GnuTLS to utilize encrypted SSL and TLS connections
\end{description}

\begin{verbatim}
brew install emacs --with-cocoa --with-imagemagick --with-gnutls
\end{verbatim}

\section{Personal Information}
\label{sec-2}

\begin{verbatim}
(setq user-full-name "Fernando Ania"
      user-mail-address "javierania21@gmail.com")
\end{verbatim}

\section{Customize settings}
\label{sec-3}

Set up the customize file to its own separate file, instead of saving
customize settings in \href{init.el}{init.el}.

\begin{verbatim}
(setq custom-file (expand-file-name "custom.el" user-emacs-directory))
(load custom-file)
\end{verbatim}

\section{Theme}
\label{sec-4}
\subsection{Atom One dark theme}
\label{sec-4-1}

\begin{verbatim}
(use-package atom-one-dark-theme
  :if (window-system)
  :ensure t
  :init 
  (progn (load-theme 'atom-one-dark t)))
\end{verbatim}

\subsection{Solarized theme}
\label{sec-4-2}

Here's some configuration for \href{https://github.com/bbatsov/solarized-emacs/}{bbatsov's solarized themes}.

\begin{verbatim}
(use-package solarized-theme
  :defer 10
  :init
  (setq solarized-use-variable-pitch nil)
  :ensure t)
\end{verbatim}

\subsection{Monokai theme}
\label{sec-4-3}

\begin{verbatim}
(use-package monokai-theme
  :if (window-system)
  :ensure t
  :init
  (setq monokai-use-variable-pitch nil))
\end{verbatim}

\subsection{Waher theme}
\label{sec-4-4}

\begin{verbatim}
(use-package waher-theme
  if (window-system)
  :ensure t
  :init
  (load-theme 'waher))
\end{verbatim}

\subsection{Convenient theme functions}
\label{sec-4-5}

\begin{verbatim}
(defun switch-theme (theme)
  "Disables any currently active themes and loads THEME."
  ;; This interactive call is taken from `load-theme'
  (interactive
   (list
    (intern (completing-read "Load custom theme: "
                             (mapc 'symbol-name
                                   (custom-available-themes))))))
  (let ((enabled-themes custom-enabled-themes))
    (mapc #'disable-theme custom-enabled-themes)
    (load-theme theme t)))

(defun disable-active-themes ()
  "Disables any currently active themes listed in `custom-enabled-themes'."
  (interactive)
  (mapc #'disable-theme custom-enabled-themes))

(bind-key "s-<f12>" 'switch-theme)
(bind-key "s-<f11>" 'disable-active-themes)
\end{verbatim}

\section{Font}
\label{sec-5}

\href{http://levien.com/type/myfonts/inconsolata.html}{Inconsolata} is a nice monospaced font.

To install it on OS X, you can use Homebrew with \href{http://caskroom.io/}{Homebrew Cask}.

\begin{verbatim}
# You may need to run these two lines if you haven't set up Homebrew
# Cask and its fonts formula.
brew install caskroom/cask/brew-cask
brew tap caskroom/fonts

brew cask install font-inconsolata
\end{verbatim}

And here's how we tell Emacs to use the font we want to use.

\begin{verbatim}
(add-to-list 'default-frame-alist
             '(font . "Inconsolata-14"))
\end{verbatim}

\section{Sane defaults}
\label{sec-6}

Let's start with some sane defaults, shall we?

Sources for this section include \href{https://github.com/magnars/.emacs.d/blob/master/settings/sane-defaults.el}{Magnars Sveen} and \href{http://pages.sachachua.com/.emacs.d/Sacha.html}{Sacha Chua}.

\begin{verbatim}
;; These functions are useful. Activate them.
(put 'downcase-region 'disabled nil)
(put 'upcase-region 'disabled nil)
(put 'narrow-to-region 'disabled nil)
(put 'dired-find-alternate-file 'disabled nil)

;; Answering just 'y' or 'n' will do
(defalias 'yes-or-no-p 'y-or-n-p)

;; Keep all backup and auto-save files in one directory
(setq backup-directory-alist '(("." . "~/.saves")))
(setq auto-save-file-name-transforms '((".*" "~/.saves/auto-save-list/" t)))

;; UTF-8 please
(setq locale-coding-system 'utf-8) ; pretty
(set-terminal-coding-system 'utf-8) ; pretty
(set-keyboard-coding-system 'utf-8) ; pretty
(set-selection-coding-system 'utf-8) ; please
(prefer-coding-system 'utf-8) ; with sugar on top
(setq-default indent-tabs-mode nil)

;; Turn off the blinking cursor
(blink-cursor-mode -1)

(setq-default indent-tabs-mode nil)
(setq-default indicate-empty-lines t)

;; Don't count two spaces after a period as the end of a sentence.
;; Just one space is needed.
(setq sentence-end-double-space nil)

;; delete the region when typing, just like as we expect nowadays.
(delete-selection-mode t)

(show-paren-mode t)

(column-number-mode t)

(global-visual-line-mode)
(diminish 'visual-line-mode)

(setq uniquify-buffer-name-style 'forward)

;; -i gets alias definitions from .bash_profile
(setq shell-command-switch "-ic")

;; Don't beep at me
(setq visible-bell t)
\end{verbatim}

The following function for \verb~occur-dwim~ is taken from \href{https://github.com/abo-abo}{Oleh Krehel} from
\href{http://oremacs.com/2015/01/26/occur-dwim/}{his blog post at (or emacs}. It takes the current region or the symbol
at point as the default value for occur.

\begin{verbatim}
(defun occur-dwim ()
  "Call `occur' with a sane default."
  (interactive)
  (push (if (region-active-p)
            (buffer-substring-no-properties
             (region-beginning)
             (region-end))
          (thing-at-point 'symbol))
        regexp-history)
  (call-interactively 'occur))

(bind-key "M-s o" 'occur-dwim)
\end{verbatim}

Here we make page-break characters look pretty, instead of appearing
as \texttt{\textasciicircum{}L} in Emacs. \href{http://ericjmritz.name/2015/08/29/using-page-breaks-in-gnu-emacs/}{Here's an informative article called "Using
Page-Breaks in GNU Emacs" by Eric J. M. Ritz.}

\begin{verbatim}
(use-package page-break-lines
  :ensure t)
\end{verbatim}

\section{Mac customizations}
\label{sec-7}

There are configurations to make when running Emacs on OS X (hence the
"darwin" system-type check).

\begin{verbatim}
(let ((is-mac (string-equal system-type "darwin")))
  (when is-mac
    ;; delete files by moving them to the trash
    (setq delete-by-moving-to-trash t)
    (setq trash-directory "~/.Trash")

    ;; Don't make new frames when opening a new file with Emacs
    (setq ns-pop-up-frames nil)

    ;; set the Fn key as the hyper key
    (setq ns-function-modifier 'hyper)

    ;; Use Command-` to switch between Emacs windows (not frames)
    (bind-key "s-`" 'other-window)

    ;; Use Command-Shift-` to switch Emacs frames in reverse
    (bind-key "s-~" (lambda() () (interactive) (other-window -1)))

    ;; Because of the keybindings above, set one for `other-frame'
    (bind-key "s-1" 'other-frame)

    ;; Fullscreen!
    (setq ns-use-native-fullscreen nil) ; Not Lion style
    (bind-key "<s-return>" 'toggle-frame-fullscreen)

    ;; buffer switching
    (bind-key "s-{" 'previous-buffer)
    (bind-key "s-}" 'next-buffer)

    ;; Compiling
    (bind-key "H-c" 'compile)
    (bind-key "H-r" 'recompile)
    (bind-key "H-s" (defun save-and-recompile () (interactive) (save-buffer) (recompile)))

    ;; disable the key that minimizes emacs to the dock because I don't
    ;; minimize my windows
    ;; (global-unset-key (kbd "C-z"))

    (defun open-dir-in-finder ()
      "Open a new Finder window to the path of the current buffer"
      (interactive)
      (shell-command "open ."))
    (bind-key "s-/" 'open-dir-in-finder)

    (defun open-dir-in-iterm ()
      "Open the current directory of the buffer in iTerm."
      (interactive)
      (let* ((iterm-app-path "/Applications/iTerm.app")
             (iterm-brew-path "/opt/homebrew-cask/Caskroom/iterm2/1.0.0/iTerm.app")
             (iterm-path (if (file-directory-p iterm-app-path)
                             iterm-app-path
                           iterm-brew-path)))
        (shell-command (concat "open -a " iterm-path " ."))))
    (bind-key "s-=" 'open-dir-in-iterm)

    ;; Not going to use these commands
    (put 'ns-print-buffer 'disabled t)
    (put 'suspend-frame 'disabled t)))
\end{verbatim}

\verb~exec-path-from-shell~ makes the command-line path with Emacs's shell
match the same one on OS X.

\begin{verbatim}
(use-package exec-path-from-shell
  :if (memq window-system '(mac ns))
  :ensure t
  :init
  (exec-path-from-shell-initialize))
\end{verbatim}

\section{List buffers}
\label{sec-8}

ibuffer is the improved version of list-buffers.

\begin{verbatim}
;; make ibuffer the default buffer lister.
(defalias 'list-buffers 'ibuffer)
\end{verbatim}


source: \url{http://ergoemacs.org/emacs/emacs_buffer_management.html}

\begin{verbatim}
(add-hook 'dired-mode-hook 'auto-revert-mode)

;; Also auto refresh dired, but be quiet about it
(setq global-auto-revert-non-file-buffers t)
(setq auto-revert-verbose nil)
\end{verbatim}

source: \href{http://whattheemacsd.com/sane-defaults.el-01.html}{Magnars Sveen}

\section{Recentf}
\label{sec-9}

\begin{verbatim}
(use-package recentf
  :commands ido-recentf-open
  :init
  (progn
    (recentf-mode t)
    (setq recentf-max-saved-items 200)

    (defun ido-recentf-open ()
      "Use `ido-completing-read' to \\[find-file] a recent file"
      (interactive)
      (if (find-file (ido-completing-read "Find recent file: " recentf-list))
          (message "Opening file...")
        (message "Aborting")))

    (bind-key "C-x C-r" 'ido-recentf-open)))
\end{verbatim}

\section{Org mode}
\label{sec-10}

Truly the way to \href{http://orgmode.org/}{live life in plain text}. I mainly use it to take
notes and save executable source blocks. I'm also starting to make use
of its agenda, timestamping, and capturing features.

It goes without saying that I also use it to manage my Emacs config.

\subsection{Org activation bindings}
\label{sec-10-1}

Set up some global key bindings that integrate with Org Mode features.

\begin{verbatim}
(bind-key "C-c l" 'org-store-link)
(bind-key "C-c c" 'org-capture)
(bind-key "C-c a" 'org-agenda)
\end{verbatim}

\subsubsection{Org agenda}
\label{sec-10-1-1}

Learned about \href{https://github.com/sachac/.emacs.d/blob/83d21e473368adb1f63e582a6595450fcd0e787c/Sacha.org#org-agenda}{this \texttt{delq} and \texttt{mapcar} trick from Sacha Chua's config}.

\begin{verbatim}
(setq org-agenda-files
      (delq nil
            (mapcar (lambda (x) (and (file-exists-p x) x))
                    '("~/Dropbox/Agenda"))))
\end{verbatim}

\subsubsection{Org capture}
\label{sec-10-1-2}

\begin{verbatim}
(bind-key "C-c c" 'org-capture)
(setq org-default-notes-file "~/Dropbox/Notes/notes.org")
\end{verbatim}

\subsection{Org setup}
\label{sec-10-2}

Speed commands are a nice and quick way to perform certain actions
while at the beginning of a heading. It's not activated by default.

See the doc for speed keys by checking out \href{(info\%20"(org)\%20speed\%20keys")}{the documentation for
speed keys in Org mode}.

\begin{verbatim}
(setq org-use-speed-commands t)
\end{verbatim}

\begin{verbatim}
(setq org-image-actual-width 550)
\end{verbatim}

\subsection{Org tags}
\label{sec-10-3}

The default value is -77, which is weird for smaller width windows.
I'd rather have the tags align horizontally with the header. 45 is a
good column number to do that.

\begin{verbatim}
(setq org-tags-column 45)
\end{verbatim}

\subsection{Org babel languages}
\label{sec-10-4}

\begin{verbatim}
(org-babel-do-load-languages
 'org-babel-load-languages
 '((python . t)
   (C . t)
   (calc . t)
   (latex . t)
   (java . t)
   (ruby . t)
   (lisp . t)
   (scheme . t)
   (sh . t)
   (sqlite . t)
   (js . t)))

(defun my-org-confirm-babel-evaluate (lang body)
  "Do not confirm evaluation for these languages."
  (not (or (string= lang "C")
           (string= lang "java")
           (string= lang "python")
           (string= lang "emacs-lisp")
           (string= lang "sqlite"))))
(setq org-confirm-babel-evaluate 'my-org-confirm-babel-evaluate)
\end{verbatim}

\subsection{Org babel/source blocks}
\label{sec-10-5}

I like to have source blocks properly syntax highlighted and with the
editing popup window staying within the same window so all the windows
don't jump around. Also, having the top and bottom trailing lines in
the block is a waste of space, so we can remove them.

I noticed that fontification doesn't work with markdown mode when the
block is indented after editing it in the org src buffer---the leading
\#s for headers don't get fontified properly because they appear as Org
comments. Setting \verb~org-src-preserve-indentation~ makes things
consistent as it doesn't pad source blocks with leading spaces.

\begin{verbatim}
(setq org-src-fontify-natively t
      org-src-window-setup 'current-window
      org-src-strip-leading-and-trailing-blank-lines t
      org-src-preserve-indentation t
      org-src-tab-acts-natively t)
\end{verbatim}

\subsection{Org exporting}
\label{sec-10-6}

\subsubsection{Pandoc exporter}
\label{sec-10-6-1}

Pandoc converts between a huge number of different file formats. 

\begin{verbatim}
(use-package ox-pandoc
  :ensure t)
\end{verbatim}

\subsubsection{org-pandoc}
\label{sec-10-6-2}

\begin{verbatim}
(use-package org-pandoc :ensure t)
\end{verbatim}

\subsection{Org bullets}
\label{sec-10-7}

\begin{verbatim}
(use-package org-bullets
 :ensure t
 :config
 (add-hook 'org-mode-hook (lambda () (org-bullets-mode 1))))
\end{verbatim}
\section{Tramp}
\label{sec-11}

\begin{verbatim}
(use-package tramp)
\end{verbatim}

\section{Locate}
\label{sec-12}

Using OS X Spotlight within Emacs by modifying the \verb~locate~ function.

I usually use \hyperref[sec-18-9]{\verb~helm-locate~}, which does live updates the spotlight
search list as you type a query.

\begin{verbatim}
;; mdfind is the command line interface to Spotlight
(setq locate-command "mdfind")
\end{verbatim}

\section{Shell}
\label{sec-13}

\begin{verbatim}
(bind-key "C-x m" 'shell)
(bind-key "C-x M" 'ansi-term)
\end{verbatim}

\section{Window}
\label{sec-14}

Convenient keybindings to resize windows.

\begin{verbatim}
(bind-key "s-C-<left>"  'shrink-window-horizontally)
(bind-key "s-C-<right>" 'enlarge-window-horizontally)
(bind-key "s-C-<down>"  'shrink-window)
(bind-key "s-C-<up>"    'enlarge-window)
\end{verbatim}

Whenever I split windows, I usually do so and also switch to the other
window as well, so might as well rebind the splitting key bindings to
do just that to reduce the repetition.

\begin{verbatim}
(defun vsplit-other-window ()
  "Splits the window vertically and switches to that window."
  (interactive)
  (split-window-vertically)
  (other-window 1 nil))
(defun hsplit-other-window ()
  "Splits the window horizontally and switches to that window."
  (interactive)
  (split-window-horizontally)
  (other-window 1 nil))

(bind-key "C-x 2" 'vsplit-other-window)
(bind-key "C-x 3" 'hsplit-other-window)
\end{verbatim}
\subsection{windmove}
\label{sec-14-1}

\begin{verbatim}
(use-package windmove :ensure t)
(windmove-default-keybindings)
(global-set-key (kbd "C-c <left>")  'windmove-left)
(global-set-key (kbd "C-c <right>") 'windmove-right)
(global-set-key (kbd "C-c <up>")    'windmove-up)
(global-set-key (kbd "C-c <down>")  'windmove-down)
\end{verbatim}
\subsection{Winner mode}
\label{sec-14-2}

Winner mode allows you to undo/redo changes to window changes in Emacs
and allows you.

\begin{verbatim}
(use-package winner
  :config
  (winner-mode t)
  :bind (("M-s-<left>" . winner-undo)
         ("M-s-<right>" . winner-redo)))
\end{verbatim}

\subsection{Transpose frame}
\label{sec-14-3}

\begin{verbatim}
(use-package transpose-frame
  :ensure t
  :bind ("H-t" . transpose-frame))
\end{verbatim}

\section{Ido}
\label{sec-15}

\begin{verbatim}
(use-package ido
  :init
  (setq ido-enable-flex-matching t)
  (setq ido-everywhere t)
  (setq ido-use-virtual-buffers t)
  (ido-mode t)
  (use-package ido-vertical-mode
    :ensure t
    :init (ido-vertical-mode 1)
    (setq ido-vertical-define-keys 'C-n-and-C-p-only)))
\end{verbatim}

\section{Whitespace mode}
\label{sec-16}

\begin{verbatim}
(use-package whitespace
  :bind ("s-<f10>" . whitespace-mode))
\end{verbatim}

\section{Google}
\label{sec-17}
\subsection{google}
\label{sec-17-1}

Emac's interface to the google api
\begin{verbatim}
(use-package google :ensure t)
\end{verbatim}

\subsection{google this}
\label{sec-17-2}

a set of functions and bindinga to google underpoint
\begin{verbatim}
(use-package google-this :ensure t)
\end{verbatim}
\section{ELPA packages}
\label{sec-18}

These are the packages that are not built into Emacs.

\subsection{Ace Jump Mode}
\label{sec-18-1}

A quick way to jump around text in buffers.

\href{http://emacsrocks.com/e10.html}{See Emacs Rocks Episode 10 for a screencast.}

\begin{verbatim}
(use-package ace-jump-mode
  :ensure t
  :diminish ace-jump-mode
  :commands ace-jump-mode
  :bind ("C-S-s" . ace-jump-mode))
\end{verbatim}
\subsection{avy}
\label{sec-18-2}

Replacing ace-window

\begin{verbatim}
(use-package avy :ensure t)
(global-set-key (kbd "C-c j") 'avy-goto-word-or-subword-1)
(global-set-key (kbd "s-.") 'avy-goto-word-or-subword-1)
(global-set-key (kbd "s-w") 'ace-window)
\end{verbatim}
\subsection{Ace Window}
\label{sec-18-3}

\href{https://github.com/abo-abo/ace-window}{ace-window} is a package that uses the same idea from ace-jump-mode for
buffer navigation, but applies it to windows. The default keys are
1-9, but it's faster to access the keys on the home row, so that's
what I have them set to (with respect to Dvorak, of course).

\begin{verbatim}
(use-package ace-window
  :ensure t
  :config
  (setq aw-keys '(?a ?o ?e ?u ?h ?t ?n ?s))
  (ace-window-display-mode)
  :bind ("s-o" . ace-window))
\end{verbatim}
\#+end$_{\text{src}}$

\subsection{Android mode}
\label{sec-18-4}

\begin{verbatim}
(use-package android-mode
  :ensure t)
\end{verbatim}

\subsection{C-Eldoc}
\label{sec-18-5}

This package displays function signatures in the mode line.

\begin{verbatim}
(use-package c-eldoc
  :commands c-turn-on-eldoc-mode
  :ensure t
  :init (add-hook 'c-mode-hook #'c-turn-on-eldoc-mode))
\end{verbatim}

\subsection{Clojure}
\label{sec-18-6}

\begin{verbatim}
(use-package clojure-mode
  :ensure t)
\end{verbatim}

\subsection{Dash}
\label{sec-18-7}

Integration with \href{http://kapeli.com/dash}{Dash, the API documentation browser on OS X}. The
binding \verb~s-D~ is the same as Cmd-Shift-D, the same binding that dash
uses in Android Studio (trying to keep things consistent with the
tools I use).

\begin{verbatim}
(use-package dash-at-point
  :ensure t
  :bind (("s-D"     . dash-at-point)
         ("C-c e"   . dash-at-point-with-docset)))
\end{verbatim}

\subsection{Highligh line}
\label{sec-18-8}

\begin{verbatim}
(use-package highlight-current-line :ensure t)
\end{verbatim}
\subsection{Helm}
\label{sec-18-9}

\begin{verbatim}
(use-package helm
  :ensure t
  :diminish helm-mode
  :init (progn
          (require 'helm-config)
          (use-package helm-projectile
            :ensure t
            :commands helm-projectile
            :bind ("C-c p h" . helm-projectile))
          (use-package helm-ag :defer 10  :ensure t)
          (setq helm-locate-command "mdfind -interpret -name %s %s"
                helm-ff-newfile-prompt-p nil
                helm-M-x-fuzzy-match t)
          (helm-mode)
          (use-package helm-swoop :ensure t :bind ("M-i" . helm-swoop)))
  :bind (("C-c h" . helm-command-prefix)
         ("C-x b" . helm-mini)
         ("C-`" . helm-resume)
         ("M-x" . helm-M-x)
         ("C-x C-f" . helm-find-files)
         ("M-<tab>" . helm-execute-persistent-action)
         ("M-y" . helm-show-kill-ring)))
\end{verbatim}

\subsection{Magit}
\label{sec-18-10}

A great interface for git projects. It's much more pleasant to use
than the git interface on the command line. Use an easy keybinding to
access magit.

\begin{verbatim}
(use-package magit
  :ensure t
  :bind ("C-c g" . magit-status)
  :config
  (define-key magit-status-mode-map (kbd "q") 'magit-quit-session))
\end{verbatim}

\subsubsection{Fullscreen magit}
\label{sec-18-10-1}

\begin{quote}
The following code makes magit-status run alone in the frame, and then
restores the old window configuration when you quit out of magit.

No more juggling windows after commiting. It's magit bliss.
\end{quote}
\href{http://whattheemacsd.com/setup-magit.el-01.html}{Source: Magnar Sveen}

\begin{verbatim}
;; full screen magit-status
(defadvice magit-status (around magit-fullscreen activate)
  (window-configuration-to-register :magit-fullscreen)
  ad-do-it
  (delete-other-windows))

(defun magit-quit-session ()
  "Restores the previous window configuration and kills the magit buffer"
  (interactive)
  (kill-buffer)
  (jump-to-register :magit-fullscreen))
\end{verbatim}

\subsection{Edit With Emacs}
\label{sec-18-11}

Editing input boxes from Chrome with Emacs. Pretty useful to keep all
significant text-writing on the web within emacs. I typically use this
with posts on Discourse, which has a post editor that overrides normal
Emacs key bindings with other functions. As such, \verb~markdown-mode~ is
used.

\begin{verbatim}
(use-package edit-server
  :ensure t
  :config
  (edit-server-start)
  (setq edit-server-default-major-mode 'markdown-mode)
  (setq edit-server-new-frame nil))
\end{verbatim}

\subsection{Elfeed}
\label{sec-18-12}

\begin{verbatim}
(use-package elfeed
  :ensure t
  :config (setq elfeed-feeds
                '("http://feeds.feedburner.com/gonintendo/news"
                  "http://usesthis.com/feed/")))
\end{verbatim}

\subsection{Emacs IPython Notebook}
\label{sec-18-13}
\begin{verbatim}
(use-package ein
  :ensure t)
\end{verbatim}

\subsection{Expand region}
\label{sec-18-14}

\begin{verbatim}
(use-package expand-region
  :ensure t
  :bind ("C-@" . er/expand-region))
\end{verbatim}

\subsection{Floobits}
\label{sec-18-15}

Using \href{https://floobits.com/}{Floobits} for code collaboration.

\begin{verbatim}
(use-package floobits
  :ensure t)
\end{verbatim}

\subsection{Flycheck}
\label{sec-18-16}

Still need to set up hooks so that flycheck automatically runs in
python mode, etc. js2-mode is already really good for the syntax
checks, so I probably don't need the jshint checks with flycheck for
it.

\begin{verbatim}
(use-package flycheck
  :ensure t
  :defer 10
  :config (setq flycheck-html-tidy-executable "tidy5"))
\end{verbatim}

\subsubsection{Linter setups}
\label{sec-18-16-1}

Install the HTML5/CSS/JavaScript linters.

\begin{verbatim}
brew tap homebrew/dupes
brew install tidy
npm install -g jshint
npm install -g csslint
\end{verbatim}

\subsection{Gists}
\label{sec-18-17}

\begin{verbatim}
(use-package gist
  :ensure t
  :commands gist-list)
\end{verbatim}

\subsection{Macrostep}
\label{sec-18-18}

Macrostep allows you to see what Elisp macros expand to. Learned about
it from the \href{https://www.youtube.com/watch?v\%3D2TSKxxYEbII}{package highlight talk for use-package}.

\begin{verbatim}
(use-package macrostep
  :ensure t
  :bind ("H-`" . macrostep-expand))
\end{verbatim}

\subsection{Markdown mode}
\label{sec-18-19}

\begin{verbatim}
(use-package markdown-mode
  :ensure t
  :mode (("\\.markdown\\'" . markdown-mode)
         ("\\.md\\'"       . markdown-mode))
         ("\\.jsx\\'" . js2-jsx-mode))
\end{verbatim}

\subsection{Multiple cursors}
\label{sec-18-20}

We'll also need to \verb~(require 'multiple-cusors)~ because of \href{https://github.com/magnars/multiple-cursors.el/issues/105}{an autoload issue}.

\begin{verbatim}
(use-package multiple-cursors
  :ensure t
  :bind (("C-S-c C-S-c" . mc/edit-lines)
         ("C->"         . mc/mark-next-like-this)
         ("C-<"         . mc/mark-previous-like-this)
         ("C-c C-<"     . mc/mark-all-like-this)
         ("C-!"         . mc/mark-next-symbol-like-this)
         ("s-d"         . mc/mark-all-dwim)))
\end{verbatim}

\subsection{Olivetti}
\label{sec-18-21}

\begin{verbatim}
(use-package olivetti
  :ensure t
  :bind ("s-<f6>" . olivetti-mode))
\end{verbatim}

\subsection{Perspective}
\label{sec-18-22}

Workspaces in Emacs.

\begin{verbatim}
(use-package perspective
  :ensure t
  :config (persp-mode))
\end{verbatim}

\subsection{Projectile}
\label{sec-18-23}

\begin{quote}
Project navigation and management library for Emacs.
\end{quote}
\url{http://batsov.com/projectile/}


\begin{verbatim}
(use-package projectile
  :ensure t
  :diminish projectile-mode
  :commands projectile-mode
  :config
  (progn
    (projectile-global-mode t)
    (setq projectile-enable-caching t)
    (use-package ag
      :commands ag
      :ensure t)))
\end{verbatim}

\subsection{Python}
\label{sec-18-24}

Integrates with IPython.

\begin{verbatim}
(use-package python-mode
  :ensure t)
\end{verbatim}

\subsection{Racket}
\label{sec-18-25}

Starting to use Racket now, mainly for programming paradigms class,
though I'm looking forward to some "REPL-driven development" whenever
I get the chance.

\begin{verbatim}
(use-package racket-mode
  :ensure t
  :commands racket-mode
  :config
  (setq racket-smart-open-bracket-enable t))

(use-package geiser
  :ensure t
  :config
  (setq geiser-default-implementation '(racket)))
\end{verbatim}

\subsection{Restclient}
\label{sec-18-26}

See \href{http://emacsrocks.com/e15.html}{Emacs Rocks! Episode 15} to learn how restclient can help out with
testing APIs from within Emacs. The HTTP calls you make in the buffer
aren't constrainted within Emacs; there's the
\texttt{restclient-copy-curl-command} to get the equivalent \texttt{curl} call
string to keep things portable.

\begin{verbatim}
(use-package restclient
  :ensure t
  :mode ("\\.restclient\\'" . restclient-mode))
\end{verbatim}

\subsection{Smartparens mode}
\label{sec-18-27}

\begin{verbatim}
(use-package smartparens
  :ensure t
  :diminish smartparens-mode
  :config
  (add-to-list 'sp--lisp-modes 'racket-mode)
  (add-to-list 'sp--lisp-modes 'geiser-mode)
  (require 'smartparens-config)
  (smartparens-global-mode t))
\end{verbatim}

\subsection{Smartparens org mode}
\label{sec-18-28}

Set up some pairings for org mode markup. These pairings won't
activate by default; they'll only apply for wrapping regions.

\begin{verbatim}
(sp-local-pair 'org-mode "~" "~" :actions '(wrap))
(sp-local-pair 'org-mode "/" "/" :actions '(wrap))
(sp-local-pair 'org-mode "*" "*" :actions '(wrap))
(add-hook 'js-mode-hook #'smartparens-mode)
\end{verbatim}

\subsection{Smartscan}
\label{sec-18-29}

\begin{quote}
Quickly jumps between other symbols found at point in Emacs.
\end{quote}
\url{http://www.masteringemacs.org/article/smart-scan-jump-symbols-buffer}


\begin{verbatim}
(use-package smartscan
  :ensure t
  :config (global-smartscan-mode 1)
  :bind (("s-n" . smartscan-symbol-go-forward)
         ("s-p" . smartscan-symbol-go-backward)))
\end{verbatim}

\subsection{Smex}
\label{sec-18-30}

Smex integrates ido with \verb~M-x~. I used to use this before moving on to
\hyperref[sec-18-9]{helm}.

\begin{verbatim}
(use-package smex
  :if (not (featurep 'helm-mode))
  :ensure t
  :bind ("M-x" . smex))
(setq smex-save-file (expand-file-name ".smex-items" user-emacs-directory))
\end{verbatim}
\subsection{Aggressive indent}
\label{sec-18-31}

\begin{verbatim}
(use-package aggressive-indent
    :ensure t)
(global-aggressive-indent-mode 1)
\end{verbatim}
\subsection{Skewer mode}
\label{sec-18-32}

Live coding for HTML/CSS/JavaScript.

\begin{verbatim}
(use-package skewer-mode
  :commands skewer-mode
  :ensure t
  :config (skewer-setup))
\end{verbatim}

\subsection{Smoothscrolling}
\label{sec-18-33}

This makes it so \verb~C-n~-ing and \verb~C-p~-ing won't make the buffer jump
around so much.

\begin{verbatim}
(use-package smooth-scrolling
  :ensure t)
\end{verbatim}

\subsection{Typescript mode}
\label{sec-18-34}

\begin{verbatim}
(use-package typescript-mode
  :ensure t)
\end{verbatim}

\subsection{Visual-regexp}
\label{sec-18-35}

\begin{verbatim}
(use-package visual-regexp
  :ensure t
  :init
  (use-package visual-regexp-steroids :ensure t)
  :bind (("C-c r" . vr/replace)
         ("C-c q" . vr/query-replace)
         ("C-c m" . vr/mc-mark) ; Need multiple cursors
         ("C-M-r" . vr/isearch-backward)
         ("C-M-s" . vr/isearch-forward)))
\end{verbatim}

\subsection{Webmode}
\label{sec-18-36}

\begin{verbatim}
(use-package web-mode
  :ensure t
  :mode (("\\.js\\'" . web-mode)
         ("\\.js\\'"       . js2-jsx-mode)
         ("\\.html\\'" . web-mode)
         ("\\.html\\.erb\\'" . web-mode)
         ("\\.mustache\\'" . web-mode)
         ("\\.jinja\\'" . web-mode)
         ("\\.php\\'" . web-mode))
  :config
  (progn
    (setq web-mode-engines-alist
          '(("\\.jinja\\'"  . "django")))))
\end{verbatim}

\subsection{Yasnippet}
\label{sec-18-37}

Yeah, snippets! I start with snippets from \href{https://github.com/AndreaCrotti/yasnippet-snippets}{Andrea Crotti's collection}
and have also modified them and added my own.

It takes a few seconds to load and I don't need them immediately when
Emacs starts up, so we can defer loading yasnippet until there's some
idle time.

\begin{verbatim}
(use-package yasnippet
  :ensure t
  :defer 10
  :diminish yas-minor-mode
  :config
  (setq yas-snippet-dirs (concat user-emacs-directory "snippets"))
  (yas-global-mode))
\end{verbatim}

\subsection{Emmet}
\label{sec-18-38}

According to \href{http://emmet.io/}{their website}, "Emmet — the essential toolkit for web-developers."

\begin{verbatim}
(use-package emmet-mode
  :ensure t
  :commands emmet-mode
  :config
  (add-hook 'html-mode-hook 'emmet-mode)
  (add-hook 'css-mode-hook 'emmet-mode))
\end{verbatim}

\subsection{Zoom-frm}
\label{sec-18-39}

\texttt{zoom-frm} is a nice package that allows you to resize the text of
entire Emacs frames (this includes text in the buffer, mode line, and
minibuffer). The \texttt{zoom-in/out} command acts similar to the
\texttt{text-scale-adjust} command---you can chain zooming in, out, or
resetting to the default size once the command has been initially
called.

Changing the \texttt{frame-zoom-font-difference} essentially enables a
"presentation mode" when calling \texttt{toggle-zoom-frame}.

\begin{verbatim}
(use-package zoom-frm
  :ensure t
  :bind (("C-M-=" . zoom-in/out)
         ("H-z"   . toggle-zoom-frame))
  :config
  (setq frame-zoom-font-difference 10))
\end{verbatim}

\subsection{Scratch}
\label{sec-18-40}

Convenient package to create \texttt{*scratch*} buffers that are based on the
current buffer's major mode. This is more convienent than manually
creating a buffer to do some scratch work or reusing the initial
\texttt{*scratch*} buffer.

\begin{verbatim}
(use-package scratch
  :ensure t)
\end{verbatim}

\subsection{Shell pop}
\label{sec-18-41}

\begin{verbatim}
(use-package shell-pop
  :ensure t
  :bind ("M-<f12>" . shell-pop))
\end{verbatim}

\subsection{SLIME}
\label{sec-18-42}

The Superior Lisp Interaction Mode for Emacs. First, Install SBCL with
brew.

\begin{verbatim}
brew install sbcl
\end{verbatim}

\begin{verbatim}
(use-package slime
  :ensure t
  :init
  (setq inferior-lisp-program "/usr/local/bin/sbcl")
  (add-to-list 'slime-contribs 'slime-fancy))
\end{verbatim}

\subsection{Quickrun}
\label{sec-18-43}

\begin{verbatim}
(use-package quickrun
  :defer 10
  :ensure t
  :bind ("H-q" . quickrun))
\end{verbatim}

\subsection{Visible mode}
\label{sec-18-44}

I found out about this mode my looking through simple.el. I use it to
see raw org-mode files without going to a different mode like
text-mode, which is what I had done in order to see invisible text
(with org hyperlinks). The entire buffer contents will be visible
while still being in org mode.

\begin{verbatim}
(use-package visible-mode
  :bind ("H-v" . visible-mode))
\end{verbatim}

\subsection{Virtualenvwrapper}
\label{sec-18-45}

\begin{verbatim}
(use-package virtualenvwrapper
  :ensure t
  :config
  (setq venv-location "~/.virtualenvs"))
\end{verbatim}
\subsection{XQuery mode}
\label{sec-18-46}

\begin{verbatim}
(use-package xquery-mode
  :ensure t)
\end{verbatim}
\subsection{\LaTeX{} Extra}
\label{sec-18-47}

\begin{verbatim}
(use-package latex-extra
  :ensure t)
\end{verbatim}

\subsection{\LaTeX{} Preview Mode}
\label{sec-18-48}

\begin{verbatim}
(use-package latex-preview-pane
  :ensure t)
\end{verbatim}

\subsection{Undo Tree}
\label{sec-18-49}

\begin{verbatim}
(use-package undo-tree
  :ensure t)
\end{verbatim}

\section{Computer-specific settings}
\label{sec-19}
\section{Languages}
\label{sec-20}
\subsection{C/Java}
\label{sec-20-1}

I don't like the default way that Emacs handles indentation. For instance,

\begin{verbatim}
int main(int argc, char *argv[])
{
  /* What's with the brace alignment? */
  if (check)
    {
    }
  return 0;
}
\end{verbatim}

\begin{verbatim}
switch (number)
    {
    case 1:
        doStuff();
        break;
    case 2:
        doStuff();
        break;
    default:
        break;
    }
\end{verbatim}

Luckily, I can modify the way Emacs formats code with this configuration.

\begin{verbatim}
(defun my-c-mode-hook ()
  (setq c-basic-offset 4)
  (c-set-offset 'substatement-open 0)   ; Curly braces alignment
  (c-set-offset 'case-label 4))         ; Switch case statements alignment

(add-hook 'c-mode-hook 'my-c-mode-hook)
(add-hook 'java-mode-hook 'my-c-mode-hook)
\end{verbatim}

\subsection{Jasmin}
\label{sec-20-2}
Mode for editing Jasmin Java bytecode assembler files.

\begin{verbatim}
(use-package jasmin)
\end{verbatim}

\subsection{Rust}
\label{sec-20-3}

\#+BEGIN$_{\text{SRC}}$ emacs-lisp
(use-package rust-mode
:ensure t)

\subsection{Wed Dev}
\label{sec-20-4}
\subsubsection{Angular}
\label{sec-20-4-1}

\begin{verbatim}
(use-package angular-mode :ensure t)
\end{verbatim}

\subsubsection{company mode}
\label{sec-20-4-2}

\begin{verbatim}
(use-package company
     :ensure t)
(add-hook 'after-init-hook 'global-company-mode)
\end{verbatim}
\subsubsection{company-web}
\label{sec-20-4-3}

\begin{verbatim}
(use-package company-web :ensure t)
\end{verbatim}
\subsubsection{auto-pair}
\label{sec-20-4-4}

\begin{verbatim}
(use-package autopair :ensure t)
;;;(autopair-global-mode 1)
\end{verbatim}
\subsubsection{Ac's}
\label{sec-20-4-5}
\begin{enumerate}
\item ac-html
\label{sec-20-4-5-1}

\begin{verbatim}
(use-package ac-html :ensure t)
\end{verbatim}
\item ac-html-angular
\label{sec-20-4-5-2}
\begin{verbatim}
(use-package ac-html-angular :ensure t)
\end{verbatim}
\item ac-html-csswatcher
\label{sec-20-4-5-3}

\begin{verbatim}
(use-package ac-html-csswatcher :ensure t)
\end{verbatim}
\end{enumerate}

\section{Misc}
\label{sec-21}
\subsection{Display Time}
\label{sec-21-1}

When displaying the time with \texttt{display-time-mode}, I don't care about
the load average.

\begin{verbatim}
(setq display-time-default-load-average nil)
\end{verbatim}

\subsection{Display Battery Mode}
\label{sec-21-2}

See the documentation for \texttt{battery-mode-line-format} for the format
characters.

\begin{verbatim}
(setq battery-mode-line-format "[%b%p%% %t]")
\end{verbatim}

\subsection{Docview keybindings}
\label{sec-21-3}

Convenience bindings to use doc-view with the arrow keys.

\begin{verbatim}
(use-package doc-view
  :commands doc-view-mode
  :config
  (define-key doc-view-mode-map (kbd "<right>") 'doc-view-next-page)
  (define-key doc-view-mode-map (kbd "<left>") 'doc-view-previous-page))
\end{verbatim}

\subsection{OS X scrolling}
\label{sec-21-4}

\begin{verbatim}
(setq mouse-wheel-scroll-amount (quote (0.01)))
\end{verbatim}

\subsection{Emacsclient}
\label{sec-21-5}

\begin{verbatim}
(use-package server
  :config
  (server-start))
\end{verbatim}

\begin{verbatim}
(use-package visible-mode
  :bind ("H-v" . visible-mode))
\end{verbatim}
\subsection{Cl-library}
\label{sec-21-6}

\begin{verbatim}
(use-package cl-lib :ensure t)
\end{verbatim}
% Emacs 25.0.91.1 (Org mode 8.2.10)
\end{document}
